\documentclass[a4paper,12pt]{article}
\usepackage{graphicx}
\usepackage{enumerate}
\usepackage{changepage} 
\usepackage{color}
\begin{document}


\title{Take Home Assignment 2019}

\author{18000768}

\maketitle

\section*{Problem}
    A giant library has just been inaugurated this week. It can be modeled as a sequence of N consecutive shelves with each shelf having some number of books. No, think of the following two queries which can be performed on these shelves.
    \begin{itemize}
        \item Change the number of books in one of the shelves
        \item Obtain the number of books on the shelf having the kth rank within the range of shelves
    \end{itemize}
    A shelf is said to have the kth rank if its position is k when the shelves are sorted based on the number of the books they contain, in ascending order. Can you write a program to simulate the above queries?
   \\ \underline{\textbf{Input Format} }
    \\The first line contains a single integer T, denoting the number of test cases. The first line of each test case contains an integer N denoting the number of shelves in the library.\\
The next line contains N space separated integers where the ith integer represents the number of books on the ith shelf where 1\textless=i\textless=N.\\
The next line contains an integer Q denoting the number of queries to be performed. Q lines follow with each line representing a query\\
Queries can be of two types:
\begin{itemize}
    \item 1 x k - Update the number of books in the xth shelf to k (1 \textless= x \textless= N).
    \item 0 x y k - Find the number of books on the shelf between the shelves x and y (both inclusive) with the kth rank (1 \textless= x \textless= y \textless= N, 1 \textless= k \textless= y-x+1).
\end{itemize}

\underline{ \textbf{Output Format} } 
\\For every test case, output the results of the queries in a new line.


    % \begin{adjustwidth}{1cm}{}
        \newpage
\section*{Algorithm}

\begin{enumerate}[Step 1:]
\item Get testcases as T
\item Set I = 0
\item Repeat 4 to 15 steps while I \textless T
% \begin{adjustwidth}{1cm}{}
\item  \hspace*{10mm} Get no of Shelves as S
\item  \hspace*{10mm} Set J = 0
\item \hspace*{10mm} Repeat 7 to 8 steps while J \textless S
\item \hspace*{15mm}    Get no of Books as B
\item \hspace*{15mm}    {\color{blue}Insert} B to linked list\\
\\ \hspace*{11mm}       [End of Loop]
\item \hspace*{10mm}Get no of queries as Q
\item \hspace*{10mm} Set J = 0
\item \hspace*{10mm}Repeat 12 to 14 steps while J \textless Q
\item \hspace*{15mm} Get query type as Qtype
\item \hspace*{15mm} if\{Qtype = 1\}\\
\hspace*{21mm}           Get x,k\\
\hspace*{21mm}          {\color{blue}Update}  xth node with k in linked list\\
\hspace*{21mm}      [End if]
\item \hspace*{15mm} if\{Qtype = 0\}\\
\hspace*{21mm}         Get x,y,k\\
\hspace*{21mm}          array = {\color{blue}Insert} data in xth to yth nodes in the linked list\\
\hspace*{21mm}         {\color{blue}Sort} array\\
\hspace*{21mm}         Print (k-1)th position in the array\\
\hspace*{16mm}      [End if]\\
\hspace*{11mm}      [End of Loop]
\item \hspace*{10mm} {\color{blue}Delete all Data} Data in linked list\\
\hspace*{1mm}  [End of Loop]



\end{enumerate}

\newpage

\subsection*{Method to Insert data to linked list}


\begin{enumerate}[Step 1:]
\item IF \{AVAIL = NULL\}\\
\hspace*{6mm}   Print OVERFLOW  \\    
\hspace*{6mm}   Go to Step 13
\\ \hspace*{1mm} [End IF]
\item set AVAIL =  AVAIL -\textgreater  next
\item newnode -\textgreater  next = NULL
\item  newnode -\textgreater  data = VAL
\item IF(start == NULL)\\
\item \hspace*{6mm}    start = newnode\\
\\ \hspace*{1mm} ELSE
\item  \hspace*{5mm} ptr = start
\item  \hspace*{5mm}Repeat step 9 while (ptr -\textgreater  next != NULL)
\item  \hspace*{15mm} ptr = ptr -\textgreater  next\\
 \\ \hspace*{5mm} [End of Loop]
\item \hspace*{5mm} ptr -\textgreater  next = newnode\\
\\ \hspace*{1mm} [End IF]
\item EXIT

\end{enumerate}

\newpage

\subsection*{Method to Update xth node with k in linked list}

\begin{enumerate}[Step 1:]

\item Get x,k

\item IF(start != NULL)
\item \hspace*{5mm} ptr = start 
\item  \hspace*{5mm} set I = 1
\item  \hspace*{5mm}Repeat 6,7 steps  while (I \textless x  AND  ptr -\textgreater  next != NULL)
\item  \hspace*{15mm} ptr = ptr -\textgreater  next
\item  \hspace*{15mm} I = I + 1\\
\\ \hspace*{5mm} [End of Loop]
\item \hspace*{5mm} ptr -\textgreater  data = k\\
\\  \hspace*{1mm} [End IF]
\item EXIT

\end{enumerate}





\subsection*{Insert data in the linked list to an array and Sort them}

\begin{enumerate}[Step 1:]

    \item Get x,y,k

    \item IF(start != NULL)
    \item \hspace*{5mm} ptr = start 
    \item  \hspace*{5mm} set I = 1
    \item  \hspace*{5mm}Repeat 6,7 steps  while (I \textless x  AND  ptr -\textgreater  next != NULL)
    \item  \hspace*{15mm} ptr = ptr -\textgreater  next
    \item  \hspace*{15mm} I = I + 1\\
    \\  \hspace*{5mm} [End of Loop]
    \item  \hspace*{5mm} set I = 0
    \item  \hspace*{5mm} int array[100]
    \item  \hspace*{5mm}Repeat 11 to 13 steps  while ( I \textless (y-x+1) )
    \item  \hspace*{15mm} array[I] = ptr -\textgreater  data
    \item  \hspace*{15mm} ptr = ptr -\textgreater  next
    \item  \hspace*{15mm} I = I + 1\\
    \\  \hspace*{5mm} [End of Loop]
    \item \hspace*{5mm} set I = 0,J = 0,Size = y-x+1
    \item \hspace*{5mm} Repeat 16 to 19 steps  while ( I \textless Size )
    \item \hspace*{15mm} Repeat 17 to 18 steps  while ( J \textless (Size - I) )
    \item \hspace*{25mm} IF (arr[I] \textgreater arr[J+i])\\
     \hspace*{36mm}         temp = arr[I]\\
     \hspace*{36mm}          arr[I] = arr[J+I]\\
     \hspace*{36mm}          arr[J+I] = temp\\
    \\ \hspace*{25mm} [End IF]
    \item \hspace*{25mm} J = J +1\\
    \\ \hspace*{15mm} [End of Loop]
    \item \hspace*{15mm} I = I +1\\
    \\ \hspace*{5mm} [End of Loop]\\
    \\\hspace*{0mm} [End IF]
    \item EXIT

\end{enumerate}

\newpage


\subsection*{Delete all data in the linked list}

\begin{enumerate}[Step 1:]

\item IF(start != NULL)

\item \hspace*{5mm} IF(start -\textgreater  next  != NULL)
\item  \hspace*{15mm} ptr = start -\textgreater  next\\
\\ \hspace*{6mm}   [End IF]
\item  \hspace*{5mm} start = NULL
\item  \hspace*{5mm}Repeat 6 to 8 steps  while ( ptr -\textgreater  next != NULL )
\item \hspace*{15mm} preptr = ptr
\item  \hspace*{15mm} ptr = ptr -\textgreater  next
\item  \hspace*{15mm} FREE(ptr)\\
\\ \hspace*{5mm} [End of Loop]\\
\\ \hspace*{1mm} [End IF]

\end{enumerate}


\end{document}
